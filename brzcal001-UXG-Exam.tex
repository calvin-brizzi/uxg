%%% template.tex
%%%
%%% This LaTeX source document can be used as the basis for your technical
%%% paper or abstract. Regardless of the length of your document, the commands
%%% are all the same.
%%% 
%%% The "\documentclass" command is the first command in your file. If you want to 
%%% prepare a version of your article with line numbers - a "review" version - 
%%% include the "review" parameter:
%%%    \documentclass[review]{acmsiggraph}
%%%

\documentclass{acmsiggraph}

%%% Title of your article or abstract.

\title{The Effect of Pop-Up Tutorials on Flow in Platformers}

\author{Calvin Brizzi\thanks{e-mail:calvin.brizzi@gmail.com}\\BRZCAL001}
\pdfauthor{Calvin Brizzi}

%%% Used by the ``review'' variation; the online ID will be printed on 
%%% every page of the content.

%\TOGonlineid{45678}

% User-generated keywords.

\keywords{flow, tutorials}

% With the "\setcopyright" command the appropriate rights management text will be added
% to your document.

\setcopyright{none}
%\setcopyright{acmcopyright}
%\setcopyright{acmlicensed}
%\setcopyright{rightsretained}
%\setcopyright{usgov}
%\setcopyright{usgovmixed}
%\setcopyright{cagov}
%\setcopyright{cagovmixed}
%\setcopyright{rightsretained}

% The year of publication in the "\copyrightyear" command.

\copyrightyear{2016}

%%% Conference information, from the completed rights management form.
%%% The "\conferenceinfo" command has two parameters: 
%%%    - conference name
%%%    - conference date and location
%%% The "\isbn" field includes the year and month after the article ISBN.

%\conferenceinfo{SIGGRAPH 2016 Posters}{July 24-28, 2016, Anaheim, CA} 
%\isbn{978-1-4503-ABCD-E/16/07} 
%\doi{http://doi.acm.org/10.1145/9999997.9999999}

\begin{document}

%%% This is the ``teaser'' command, which puts an figure, centered, below 
%%% the title and author information, and above the body of the content.

 %\teaser{
 %  \includegraphics[height=1.5in]{images/sampleteaser}
 %  \caption{Spring Training 2009, Peoria, AZ.}
 %}

\maketitle

\begin{abstract}

We present the results from an experiment (n=10) which uses a standard measure of flow to measure the effect of pop-up tutorials on player experience.\\
The results, while not statistically significant, suggest that pop-up tutorials may have a positive effect on flow but further studies with a larger experiment size are needed.

\end{abstract}

%
% The code below should be generated by the tool at
% http://dl.acm.org/ccs.cfm
% Please copy and paste the code instead of the example below. 
%
\begin{CCSXML}
<ccs2012>
<concept>
<concept_id>10003120.10003121.10003122.10003334</concept_id>
<concept_desc>Human-centered computing~User studies</concept_desc>
<concept_significance>300</concept_significance>
</concept>
</ccs2012>
\end{CCSXML}

\ccsdesc[300]{Human-centered computing~User studies}

%
% End generated code
%

% The next three commands are required, and insert the user-generated keywords, 
% The CCS concepts list, and the rights management text.
% Please make sure there is a blank line between each of these three commands.

\keywordlist

\conceptlist

\printcopyright

\section{Introduction}

Ut sagittis arcu ut turpis sodales, nec venenatis magna efficitur. Fusce non rhoncus risus, ac tincidunt arcu. Nulla lacus odio, accumsan tempor dolor sit amet, tincidunt porttitor justo. Quisque vulputate ex ac purus ultrices tristique. Pellentesque habitant morbi tristique senectus et netus et malesuada fames ac turpis egestas. Curabitur sed ullamcorper metus. Phasellus eu purus eget leo vulputate auctor vel scelerisque velit.

\begin{table}[ht]
  \centering
  \caption{A simple table.}
  \begin{tabular}{|r|l|}
    \hline
    7C0 & hexadecimal \\
    3700 & octal \\ \cline{2-2}
    11111000000 & binary \\
    \hline \hline
    1984 & decimal \\
    \hline
  \end{tabular}
\end{table}
  
Etiam sed mattis justo. Mauris lorem sapien, pellentesque vel viverra varius, porta ut nisi. Cras vel interdum dui, vitae fermentum elit. Nulla eu libero finibus, bibendum elit nec, ullamcorper velit. Donec ultrices, purus id ullamcorper euismod, ipsum erat sodales augue, ut sagittis sapien magna nec ex. Nulla massa arcu, suscipit non molestie ut, tristique id tellus. Maecenas nec malesuada mauris, vitae mattis sem. Quisque at risus quis arcu eleifend lacinia non sed neque.

\section{Experiment}

We conducted this experiment as part of the User Experience in Games module of the Computer Science Honours Degree at University of Cape Town. A direct comparison of flow scores collected from two separate groups of users each playing a slightly different version of the same game was made: one group acted as a control and played a version where the tutorial prompts were part of the game's background, incorporated into the level, the other group played a version of the game where at specific intervals a pop-up screen would interrupt the game with instructions for the player.\\
The purpose of the experiment was to investigate the effect the pop-up tutorial would have on flow as measured by the Short Flow State Scale by Jackson [TO DO REFERENCE!!] which was selected due to it being the closest to a standard for flow measurement.

\subsection{Participants}

We recruited 10 unpaid volunteers who were mostly second and third year computer science students. We did not collect demographic information. Participants were initially randomly assigned to one of the two conditions until the final participants who were assigned to maintain an even number of players for each version.\\
Before beginning they each confirmed that they'd never played the game before (as a version is freely available online).

\subsection{Game}

The game used was Squishy Block, a simple 2D platformer. The game was modified to have two versions: a background tutorial version and a pop-up tutorial version.
In the background version, information on how to play was displayed on small signs that were incorporated into the level. In the pop-up version, information on how to play was displayed in full-screen pop-ups that needed to be manually dismissed by the player before the game could be resumed.

It is important to note that the game was identical in every other way, the text displayed in the pop-up and background prompts was the same and the pop-up prompts triggered at the same point that the background signs were visible to the player to ensure any differences in experience were caused by the format of the tutorial and not the contents. The different versions also had identical soundtrack and sound effects.

The game consisted of two levels; the first contained the tutorial which intruduced the player to the controls and concepts of the game gradually, just before those controls or concepts were necessary (explaining how to deal with enemies just before the first enemy was encountered for example), the second level had no tutorial elements and was designed in such a way to test the player on everything introduced in the first level.

\subsection{Procedure}
The experiment was conducted in three phases: an initial brief questionnaire, the game itself and then a final questionnaire.\\
In the initial phase subjects were briefed on how the experiment would be conducted and asked to rate their previous experience with platformers. Then they switched over to the game, played through the two levels (or until their character died three times) and finally switched back to the final questionnaire, which was the Short Flow State Scale by Jackson.\\
The players were given no instructions on how to play the game before beginning (as this was left up to the tutorial).


%\begin{figure}[ht]
%  \centering
%  \includegraphics[width=3.0in]{images/ferrari_laferrari}
%  \caption{Ferrari LaFerrari. Image courtesy Flickr user ``gfreeman23.''}
%  \label{fig:ferrari}
%\end{figure}

\section{Third Section Heading}

Ut sagittis arcu ut turpis sodales, nec venenatis magna efficitur. Fusce non rhoncus risus, ac tincidunt arcu. Nulla lacus odio, accumsan tempor dolor sit amet, tincidunt porttitor justo. Quisque vulputate ex ac purus ultrices tristique. Pellentesque habitant morbi tristique senectus et netus et malesuada fames ac turpis egestas. Curabitur sed ullamcorper metus. Phasellus eu purus eget leo vulputate auctor vel scelerisque velit.

Nunc vitae lorem nec diam ultrices fringilla. Aliquam volutpat metus ut magna bibendum, sed ultricies nunc placerat. Nulla volutpat rutrum vehicula. Cum sociis natoque penatibus et magnis dis parturient montes, nascetur ridiculus mus. Aliquam vel ligula elit. Nulla fermentum purus eu venenatis mollis. Nulla placerat dui accumsan urna pharetra maximus. Sed nec orci arcu. Suspendisse faucibus blandit libero ut feugiat. Nulla vitae imperdiet nulla. Cum sociis natoque penatibus et magnis dis parturient montes, nascetur ridiculus mus.

Etiam sed mattis justo. Mauris lorem sapien, pellentesque vel viverra varius, porta ut nisi. Cras vel interdum dui, vitae fermentum elit. Nulla eu libero finibus, bibendum elit nec, ullamcorper velit. Donec ultrices, purus id ullamcorper euismod, ipsum erat sodales augue, ut sagittis sapien magna nec ex. Nulla massa arcu, suscipit non molestie ut, tristique id tellus. Maecenas nec malesuada mauris, vitae mattis sem. Quisque at risus quis arcu eleifend lacinia non sed neque.

\section*{Acknowledgements}

To Robert, for all the bagels.

\bibliographystyle{acmsiggraph}
\nocite{*}
\bibliography{template}
\end{document}
